\documentclass[12pt,letterpaper]{article}
\usepackage[letterpaper,margin=0.75in]{geometry}
\usepackage[utf8]{inputenc}
\usepackage{mdwlist}
\usepackage[T1]{fontenc}
\usepackage{textcomp}
\usepackage{tgpagella}
\usepackage{multicol}
\usepackage{fancyhdr}
\usepackage{lastpage}

\pagestyle{empty}
\setlength{\tabcolsep}{0em}

%\pagestyle{fancy}
\fancyhf{}
\lhead{}
\chead{}
\rhead{}
\rfoot{\thepage/ \pageref{LastPage}}
\lfoot{Jason Bane}
\cfoot{jbane@jlab.org}

% indentsection style, used for sections that aren't already in lists
% that need indentation to the level of all text in the document
\newenvironment{indentsection}[1]%
{\begin{list}{}%
		{\setlength{\leftmargin}{#1}}%
		\item[]%
	}
	{\end{list}}

% opposite of above; bump a section back toward the left margin
\newenvironment{unindentsection}[1]%
{\begin{list}{}%
		{\setlength{\leftmargin}{-0.5#1}}%
		\item[]%
	}
	{\end{list}}

% format two pieces of text, one left aligned and one right aligned
\newcommand{\headerrow}[2]
{\begin{tabular*}{\linewidth}{l@{\extracolsep{\fill}}r}
		#1 &
		#2 \\
\end{tabular*}}

% make "C++" look pretty when used in text by touching up the plus signs
\newcommand{\CPP}
{C\nolinebreak[4]\hspace{-.05em}\raisebox{.22ex}{\footnotesize\bf ++}}

% and the actual content starts here
\begin{document}

\vspace*{-1.2cm}
\noindent Jason Bane\\
18 Morrison Ave. \\
Newport News, VA 23601 \\
jbane@jlab.org \\
\today\\

Dear Hiring Manager,

\paragraph{}I write to apply for the Postdoctoral Research Associate - Neutrinos and Advanced Detectors (2275). I successfully defended my dissertation "The EMC Effect in A=3 Nuclei" under the direction of Nadia Fomin, at the University of Tennessee. My dissertation research brought me to Thomas Jefferson National Laboratory in Newport News, VA. My time at the lab has been devoted to honing my skills and knowledge in the complete process of being an experimental physicist. I experienced constructing spectrometers, refurbishing detectors, building electrical systems for data acquisition, analyzing large data sets, and collaborating with a diverse community while working with three electron scattering experimental groups.
\paragraph{}My research at Jefferson Lab has allowed me to work on many aspects of an experiment. I collaborated with a diverse group of scientists to construct, maintain, and operate three electron spectrometers to extract the electron-nucleus inclusive cross section. 
My hands on experience consisted of refurbishing an electron spectrometer, including repairing Cherenkov PMTs, scintillator bars, and building front end electronics. A large portion of my research has consisted of extracting the experimental measured cross section from electron scattering data. Using this data, I measured the performance of the individual detectors by controlling data sampling through signal cuts. I have had to remove background events using fits and comparisons. I used Monte Carlo simulations to study the acceptance probability of the spectrometer. These experiences in data analysis have allowed me to learn many different coding techniques and languages like Python, \CPP, Fortran, ROOT, and Java. All of these analysis task have required me to think critical about the task at hand, and I have had to adapt my problem solving skills to deal with solving algorithms, studying uncertainties, and debugging my code and the code of others. 


\paragraph{}The knowledge and skills I have gained working at Jefferson Lab would make me a great fit to work as a postdoctoral research associate at ORNL. My mixture of hardware and software experiences will allow me to quickly step into the roles needed.      
\\

\noindent Thank you for your time and attention,\\
\noindent Jason Bane




\end{document}

