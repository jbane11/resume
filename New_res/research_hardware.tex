
\documentclass[12pt,letterpaper]{article}
\usepackage[letterpaper,margin=0.75in]{geometry}
\usepackage[utf8]{inputenc}
\usepackage{mdwlist}
\usepackage[T1]{fontenc}
\usepackage{textcomp}
\usepackage{tgpagella}
\usepackage{multicol}
\usepackage{fancyhdr}

\pagestyle{empty}
\setlength{\tabcolsep}{0em}

\pagestyle{fancy}
\fancyhf{}
\lhead{}
\chead{}
\rhead{}
\rfoot{\thepage}
\lfoot{Jason Bane}
\cfoot{jbane@jlab.org}

% indentsection style, used for sections that aren't already in lists
% that need indentation to the level of all text in the document
\newenvironment{indentsection}[1]%
{\begin{list}{}%
	{\setlength{\leftmargin}{#1}}%
	\item[]%
}
{\end{list}}

% opposite of above; bump a section back toward the left margin
\newenvironment{unindentsection}[1]%
{\begin{list}{}%
	{\setlength{\leftmargin}{-0.5#1}}%
	\item[]%
}
{\end{list}}

% format two pieces of text, one left aligned and one right aligned
\newcommand{\headerrow}[2]
{\begin{tabular*}{\linewidth}{l@{\extracolsep{\fill}}r}
	#1 &
	#2 \\
\end{tabular*}}

% make "C++" look pretty when used in text by touching up the plus signs
\newcommand{\CPP}
{C\nolinebreak[4]\hspace{-.05em}\raisebox{.22ex}{\footnotesize\bf ++}}

% and the actual content starts here
\begin{document}

{\textbf{Jason Bane}}\\
18 Morrison Ave. \\
Newport News, VA 23601 \\
(931) 239-0611 \\
jbane@jlab.org \\
\today\\


\subsection*{\centerline{Research Statement}}

\paragraph{}
I completed my Ph.D. research at Thomas Jefferson Notional Laboratory through the University of Tennessee advised by Nadia Fomin. My research brought me from Knoxville to Newport News to gain access to Jefferson Lab's electron accelerator. I used an electron to probe inside of a nucleus to gain a greater understanding of the building blocks of nature and their internal interactions.

\subsection*{Current Research}
\paragraph{}The main focus of my current research is to study the internal structure of a nucleon using electron scattering. Specifically, this research is trying to study the EMC effect in both the Helium-3 and Tritium nuclei from data that was taken during the MARATHON(MeAsurement of $F_n^2/F_p^2$, d/u RAtios and A=3 EMC Effect in Deep Inelastic Electron Scattering off the Tritium and Helium MirrOr Nuclei) experiment. This research is trying to discover how the modification of the tritium nuclear structure function relative to the structure function of deuterium differs from the modification of the Helium nuclear structure function relative to the structure function of deuterium. The mirror nuclear structure of tritium and helium-3 will allow the extraction of the modification caused by the additional proton in helium-3 versus the additional neutron in tritium. 
\paragraph{} In order to study the EMC effect in tritium and helium-3, the deep inelastic cross section has to be measured for both of the mirror nuclei and deuterium. My thesis work covered measuring the deep inelastic cross section of tritium, helium-3, and deuterium at an x-Bjorken(x) between 0.2 and 0.8. The cross section can be measured experimentally by counting the number of scattered electrons versus the number of incident electrons and the number of scattering centers. This can be constrained to a double differential equation when using a limited phase space detector, and can be written as:
\begin{center}
$ \frac{d\sigma}{d\Omega dE^\prime} = \frac{Ne - BG }{Luminosity *  \epsilon * Acc(E^\prime ,\theta) * \Delta\Omega,\Delta E^\prime)} . $
 \end{center}
\paragraph{}During this experiment, I used the spectrometers to count scattered electrons($N_e$). Our spectrometer cannot determine the exact origination and type of scattering interaction of the electron, so I must subtract background events($BG$). The spectrometers are designed to detect leptons or hadrons scattering from the target chamber, but they are not absolutely efficient. In order to correct for the inefficiency, I apply an efficiency correction ($\epsilon$). The acceptance of the spectrometers is not uniform over the entire phase space. An acceptance function $Acc(E^\prime ,\theta)$ corrects for lost events. The high resolution spectrometers can only accept scattering events in a small phase space in angle and scattered momentum $ (\Delta\Omega,\Delta E^\prime)$. 
\paragraph{}Instead of calculating the acceptance function and phase space of the spectrometers, I used a monte carlo simulation to compare with data to calculate the cross section. If you generate the events in the same phase space with the same acceptance, the yield of data can be associated with the yield from data with:
\begin{center}
$ Yield_{data} = \textit{L} *\sigma^{data} * \left( \epsilon*\Delta E^\prime \Delta \Omega\right)*  A \left(E^\prime \theta \right)$\\
$ Yield_{MC} = \textit{L} *\sigma^{mod} * \left( \Delta E^\prime \Delta \Omega\right)*  A \left(E^\prime \theta \right)$\\

$ \frac{d\sigma}{d\Omega dE^\prime} = \sigma^{mod} * \left[\frac{Yield_{data} \left( 
	E^\prime,\theta\right)} {Yield_{MC}\left(E^\prime,\theta\right)}\right] $
\end{center}
\paragraph{}
The analysis process of extracting the cross section has focused on the calculation of the correction and weighting factors for the measurement of the cross section. The efficiency correction is a combination of inefficiencies from the detectors such as dead time and trigger efficiency, and software efficiencies as tracking and particle identification. The spectrometer's overall efficiency differs with kinematic, mainly due to electron and pion rate. The luminosity of a kinematic is calculated with the total charge accumulated over that kinematic and the density of the chosen nuclear target. Due to heating of the target gas from the indecent electron beam, the density of the target gas can change. This density change has been calculated for each of the gaseous targets, and the effect can be as large as 10$\%$. 
\paragraph{}Currently, I am in the process of finalizing the systematic studies of the corrections for background events and monte carlo comparison. The background events for the MARATHON experiment are dominated by two different originators, pair produced electron-positron pairs and electrons that scatter off of the aluminum end caps. High energy photons decay into an e$^+$e$^-$ pairs. The electrons from these pairs can be scattered into our spectrometers, but cannot be distinguished via our detectors. A correction for these background events is calculated by measuring the number of positrons seen at the same kinematics. The background events that scatter from the aluminum end caps are cut out via an acceptance cut in the reaction vertex along the beam line. After studying yields from the empty cell, there is a non negligible amount of events that scatter from the end caps that tail off into accepted region of the reaction vertex. These background events are corrected by extracting the yield of events from the empty cell and  normalizing that to the events from the end caps of the gas cells. 
\paragraph{}The next step in my analysis process of calculating the cross section for my three gaseous targets is to understand the systematic uncertainty in the comparison of monte carlo to data. I will begin to look at how adjusting small features of the event generation like offsets, optics matrix, generation phase space, and magnet apertures will effect the comparison of data to monte carlo. Then I will use these small changes in comparison to accurately describe the uncertainty of the monte carlo to data ratio.  
\subsection*{Future Research}
\paragraph{}My time at Jefferson Lab has taught many many aspects of the process of inclusive cross section measurement from electron scattering. I want to take this knowledge and advance my own understanding of not only inclusive electron scattering but also other scattering processes. I want to use the tools that I have learned here to develop a more complete interpretation of the structure and interactions of the fundamental aspects of nature. I thinking joining the Center for Neutrino Physics at Virginia Tech to develop and test nuclear models for neutrino interactions would be a great first step in achieving a greater understanding of the structure and interactions of the fundamental aspects of nature. 



\newpage
-
\newpage
{\textbf{Jason Bane}}\\
18 Morrison Ave. \\
Newport News, VA 23601 \\
(931) 239-0611 \\
jbane@jlab.org \\
\today\\

\subsection*{\centerline{Hardware Experiences}}

\paragraph{} My work at Jefferson Lab since 2014 has allowed me to gain hardware experiences in three different areas. In 2014, I worked on refurbishing and setting up the Big Bite large acceptance electron spectrometer. During the spring and summer 2017, I was able to work with the Hall A beamline detectors. Then in the fall of 2017, I had experience working with the two hall A high resolution spectrometers.
\subsection*{Big Bite}
\paragraph{}The original proposal for the MARATHON experiment was to use the Big Bite electron spectrometer to increase the rate of electron counting to achieve better statics at low rate kinematics. In order to use Big Bite, the spectrometer needed to be reassembled, reconnected with power and signal cables, and tested. The spectrometer consisted of a gas cherenkov with 20 photomultiplier tubes(PMT), two wire chambers, a plane of scintillators with 13 bars, and two lead glass calorimeters with 143 blocks. The hardware experience I received during my work with the Big Bite spectrometer can be divided into refurbishing the gas cherenkov, performing maintenance on the wire chambers, reconnecting power and signal cables, and testing electronic modules.
\begin{itemize*}
	\item Refurbishing the gas cherenkov
	\begin{itemize*}
		\item Testing and replacing malfunctioning PMTs
		\item Leak checking the chamber
		\item Aligning and cleaning mirror panels.
	\end{itemize*}
	\item Maintaining the wire chambers
	\begin{itemize*}
		\item Test and adjust signal electronic cards
		\item Clean and test wires with increasing voltages. 
	\end{itemize*}
	\item Connecting power and signal cables
	\begin{itemize*}
		\item Measuring the length of long cables with an oscilloscope, testing, and labeling signal and high voltage cables
		\item Repairing broken cables and creating new ribbon cables. 
		\item Design cable layout for trigger and data acquisition.
		\item Design high voltage cable layout including mapping high voltage crate.
	\end{itemize*}
	\item Testing electronics
	\begin{itemize*}
		\item Testing high voltage power supply cards
		\item Testing level translators for the wire chambers
		\item Testing ADC and TDC cards in a fastbus crate with singles from a pulser and cosmic signals. 
	\end{itemize*}
\end{itemize*}

 \subsection*{Hall A Beam Line}
\paragraph{} During the spring run period of 2017, the Argon experiment saw some weird unexpected signals from the beam position monitors(BPM) on the hall a beam line. I was charged with investigating this issue to prepare for the MARATHON experiment. My examination of the signal issue required me to search through the signal pathway of the BPM, from the BPMs to the ADC channels. I was required to test signals from the BPM throughout their journey. This included testing RF modules, amplifiers, digital to analog converters, multiplexers, and demultiplexers. My investigation discovered a faulty fan was causing the demultiplexer to fail. After replacing the fan, tests were completed to ensure the signal quality of the BPMs. 
  
\subsection*{Hall A High Resolution Spectrometers(HRS)}
\paragraph{}The Tritium group of experiments were slated to begin late fall of 2017. In order to prepare for this group of experiments, I performed maintenance on the hall A HRSs. The preparation for the upcoming experiments consists of many routine task. Some of those include testing and replacing fault cables, adjusting and testing the timing for the scintillators, testing ADC and TDC channels and the signals associated with those channels, installing new type of ADCs, building the trigger logic, and correcting the timing of the triggering signals. 




\end{document}