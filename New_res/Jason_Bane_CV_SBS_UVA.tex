% resume.tex
% vim:set ft=tex spell:

\documentclass[12pt,letterpaper]{article}
\usepackage[letterpaper,margin=0.75in]{geometry}
\usepackage[utf8]{inputenc}
\usepackage{mdwlist}
\usepackage[T1]{fontenc}
\usepackage{textcomp}
\usepackage{tgpagella}
\usepackage{multicol}
\usepackage{fancyhdr}
\usepackage{lastpage}

\pagestyle{empty}
\setlength{\tabcolsep}{0em}

\pagestyle{fancy}
\fancyhf{}
\lhead{}
\chead{}
\rhead{}
\rfoot{\thepage/ \pageref{LastPage}}
\lfoot{Jason Bane}
\cfoot{jbane@jlab.org}

% indentsection style, used for sections that aren't already in lists
% that need indentation to the level of all text in the document
\newenvironment{indentsection}[1]%
{\begin{list}{}%
	{\setlength{\leftmargin}{#1}}%
	\item[]%
}
{\end{list}}

% opposite of above; bump a section back toward the left margin
\newenvironment{unindentsection}[1]%
{\begin{list}{}%
	{\setlength{\leftmargin}{-0.5#1}}%
	\item[]%
}
{\end{list}}

% format two pieces of text, one left aligned and one right aligned
\newcommand{\headerrow}[2]
{\begin{tabular*}{\linewidth}{l@{\extracolsep{\fill}}r}
	#1 &
	#2 \\
\end{tabular*}}

% make "C++" look pretty when used in text by touching up the plus signs
\newcommand{\CPP}
{C\nolinebreak[4]\hspace{-.05em}\raisebox{.22ex}{\footnotesize\bf ++}}

% and the actual content starts here
\begin{document}

\begin{center}
{\LARGE \textbf{Jason Bane}}

18 Morrison Ave.\ \ \textbullet
\ \ Newport News, VA 23601 \\
(931) 239-0611\ \ \textbullet
\ \ jbane@jlab.org
\end{center}

\hrule
\vspace{-0.4em}
\subsection*{Education}

\begin{itemize*}
	\parskip=0.1em
	
	\item 
	\headerrow
	{\textbf{University of Tennessee}}
	{\textbf{Knoxville, TN}}
	
	\headerrow
	{\emph{Ph.D. in Nuclear Physics}}
	{\emph{August 2012 -- Planned December 2019}}
		\headerrow
	{\emph{Thesis: The EMC Effect in A=3 Nuclei}}
	{\emph{Advisor: Nadia Fomin}}
	%	\begin{itemize*}
	%		\item Lorem ipsum dolor sit amet, consectetuer adipiscing elit.
	%		\item Mirum est notare quam littera gothica, quam nunc putamus parum
	%		claram.
	%	\end{itemize*}
	
	
	\parskip=0.1em
	
	\item 
	\headerrow
	{\textbf{University of Tennessee}}
	{\textbf{Knoxville, TN}}
	\\
	\headerrow
	{\emph{Secondary Education Certification in Math and Science}}
	{\emph{August 2009 -- May 2010}}
	%	\begin{itemize*}
	%		\item Lorem ipsum dolor sit amet, consectetuer adipiscing elit.
	%		\item Mirum est notare quam littera gothica, quam nunc putamus parum
	%		claram.
	%	\end{itemize*}
	
	
	\parskip=0.1em
	
	\item 
	\headerrow
	{\textbf{University of Tennessee}}
	{\textbf{Knoxville, TN}}
	\\
	\headerrow
	{\emph{Bachelor of Science, Physics \& Minor in Education }}
	{\emph{August 2004 -- May 2009}}
	
	%	\begin{itemize*}
	%		\item Lorem ipsum dolor sit amet, consectetuer adipiscing elit.
	%		\item Mirum est notare quam littera gothica, quam nunc putamus parum
	%		claram.
	%	\end{itemize*}
	
\end{itemize*}
\hrule
\vspace{-0.4em}
\subsection*{Experience}

\begin{itemize}
	\parskip=0.1em

	\item
	\headerrow
		{\textbf{University of Tennessee, Department of Physics and Astronomy }}
		{\textbf{  Knoxville, TN,}}
	\\
	\headerrow
		{\emph{Graduate Research Assistant }}
		{\emph{May 2014 -- Present}}
	\begin{itemize}
			\item Designed and constructed front end electronics for an electron spectrometer.
			\item Created module layouts and cable maps for efficient reuse of products.
			\item Tested high voltage cards and laid high voltage cable for an electron spectrometer.
			\item Used Oscilloscopes to test signals, debug logic modules, and map out inconsistent signals.  
			\item Maintained and refurbished individual detector components of a spectrometer including checking the quality of Photo Multiplier Tubes and plastic scintillators.
			\item Calibrated detectors and used online analysis tools in Java to control the quality of data during an experiment. 
			\item Performed analysis on a large set of data involving multiple nuclear targets using Python, C++ , ROOT, and fortran. 
			\item Instructed new researchers on the use of hardware and software used in the field
	\end{itemize}
	\vspace{1cm}
	\item
	\headerrow
		{\textbf{University of Tennessee, Department of Physics and Astronomy }}
		{\textbf{  Knoxville, TN,}}
	\\
	\headerrow
		{\emph{Graduate Teaching Assistant }}
		{\emph{August 2012 -- May 2015}}
	\begin{itemize}
		\item Designed and implemented observational and planetarium based astronomy labs.
		\item Educated students on the use of refracting telescopes and equatorial mounts.
		\item Instructed students in laboratory exercises to help conceptualize physics topics.
		\item Tutored students for homework assistance and test prep.
	\end{itemize}
	\begin{samepage}
	\item
	\headerrow
		{\textbf{Clay County Tennessee Education Department}}
		{\textbf{Celina, TN}}
	\\
	\headerrow
		{\emph{Secondary Educator \& Football Coach}}
		{\emph{August 2010 -- May 2012 }}
	\begin{itemize*}
		\item Created lesson plans that included interactive, creative thinking, and discussion driven curriculum for a diverse body of geometry students.
		\item Constructed lessons that used hands-on lab activities, demonstrations, and interactive computer lessons to instruct high school Juniors and Seniors in algebra-based physics. 
		\item Used discussion-based problem-solving lessons to help remedial math students to improve their algebra, geometry and trigonometry skills for post-secondary education. 
		\item Provided an equitable and inclusive atmosphere for diverse students. 
		\item Math and reading focused tutoring. 
	\end{itemize*}
	\end{samepage}
\end{itemize}




\hrule
\vspace{-0.4em}
\subsection*{Core Technical Skills}

\begin{indentsection}{\parindent}
\hyphenpenalty=1000
\begin{description*}
	\item[Hardware:]
	Detector maintenance and wiring, front end electronics design and implementation, logical trigger design and testing   
	\item[Languages:]
	C, \CPP, \LaTeX, Python, shell script, SQL \\
	Monte Carlo Simulation Packages\\
	Example scripts located at https://github.com/jbane11/examples
	\item[Software:]
	Microsoft Office, Libre Office, Texstudio, vim, atom
	\item[Operating Systems:]
	Linux(Red Hat), Windows, MacOS
	
\end{description*}
\end{indentsection}
\hrule
\subsection*{Publications}
\begin{itemize} \itemsep -2pt % Reduce space between items
	\item  H. Dai, [et al. including \textbf{J. Bane}], ``First Measurement of the Ar(e,$e^\prime$)X Cross Section at Jefferson Lab," Phys. Rev. C 99, 054608 May 2019
	\item R. Cruz-Torres, [et al. including \textbf{J. Bane}], ``Comparing proton momentum distributions in A=3 nuclei via $^3$He and $^3H$(e,$e^\prime$p) measurements," in preparation, (2019)
	\item S. N. Santiesteban, S. Alsalmi, D. Meekins, \textbf{J. Bane,} et al., ``Density Changes in Low Pressure Gas Targets for Electron Scattering Experiments" NIM A 940, 2019
	\item  H. Dai, [et al. including \textbf{J. Bane}], ``First Measurement of the Ti(e,$e^\prime$)X Cross Section at Jefferson Lab," Phys. Rev. C 98, 014617 July 2018
	\item P V. Pandey, [et al. including \textbf{J. Bane}], ``Probing electron-argon scattering for liquid-argon based neutrino-oscillation program," preprint arXiv:1711.01671
	
\end{itemize}
\hrule
\subsection*{Honors}
	\begin{itemize*}
		\item Jefferson Science Associates graduate fellowship award (2018)  
		\item Chancellor’s honors for extraordinary professional promise (2016) 
		\item DOE Office of Science Graduate Student Research program award (2015)
		\item Dean's List 2009 Academic Year (2010)
	\end{itemize*}
\hrule
\subsection*{Conference Presentations}
\begin{itemize*}
	\item "EMC in A=3 from MARATHON," 2nd Workshop on Quantitative Challenges in SRC and EMC Research,  MIT, Cambridge MA,  March 2019
	\item "Ratios in A=3 nuclei from MARATHON ," American Physical Society's Division of Nuclear Physics' yearly meeting, HA, October 2018
	\item "Measurement of the spectral function of Argon and Titanium through the(e,$e^\prime$p) reaction," American Physical Society's Division of Nuclear Physics' yearly meeting, HA, October 2018
	\item "Status of the MARATHON experiment." American Physical Society's Division of Nuclear Physics' yearly meeting, Pittsburgh PA, October 2017
	\item "Searching for the Origin of the EMC effect." American Physical Society's Division of Nuclear Physics' yearly meeting, Sante Fe NM, October 2016
\end{itemize*}

\hrule
\subsection*{Poster Presentations}
\begin{itemize*}
	\item "The impetus in the EMC effect, a EMC simulation." Gordon Research Conferences, Holderness, NH. August 2018 
	\item "Searching for the Origin of the EMC effect." SURA Board of Trustees Meeting, Newport News, VA. April 2018 
\end{itemize*}
\hrule
\subsection*{References}
\begin{multicols}{2}
	{\noindent\bf{Nadia Fomin}}, Professor\\
	Department of Physics and Astronomy\\ 
	University of Tennessee at Knoxville\\
	(865) 974-1509, nfomin@utk.edu\\
	\columnbreak
	
	{\noindent\bf{Cynthia Keppel}}, Hall A and C Leader\\
	Jefferson Lab Accelerator Facility\\
	(757) 584-7580, keppel@jlab.edu\\
\end{multicols}
%\begin{multicols}{1}
	{\noindent\bf{Douglas Higinbotham}}, Staff Scientist\\
	Jefferson Lab Accelerator Facility\\
	(757) 584-7851, doug@jlab.edu\\

%	\columnbreak	

%	{\noindent\bf{Evan McClellan}}, Post-doctoral Fellow\\
%	Jefferson Lab Accelerator Facility\\
%	randallm@jlab.org\\
%\end{multicols}


\hrule

\subsection*{Interests}

	Football, coaching, programming, boating, traveling,


\end{document}


  
%	{\bf{Mellisa White}}, Teacher, Principal\\
%Clay County Board of Education\\
%millicentfiske@gmail.com\\	

