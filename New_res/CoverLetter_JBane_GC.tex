\documentclass[12pt,letterpaper]{article}
\usepackage[letterpaper,margin=0.75in]{geometry}
\usepackage[utf8]{inputenc}
\usepackage{mdwlist}
\usepackage[T1]{fontenc}
\usepackage{textcomp}
\usepackage{tgpagella}
\usepackage{multicol}
\usepackage{fancyhdr}
\usepackage{lastpage}

\pagestyle{empty}
\setlength{\tabcolsep}{0em}

%\pagestyle{fancy}
\fancyhf{}
\lhead{}
\chead{}
\rhead{}
\rfoot{\thepage/ \pageref{LastPage}}
\lfoot{Jason Bane}
\cfoot{jbane@jlab.org}

% indentsection style, used for sections that aren't already in lists
% that need indentation to the level of all text in the document
\newenvironment{indentsection}[1]%
{\begin{list}{}%
		{\setlength{\leftmargin}{#1}}%
		\item[]%
	}
	{\end{list}}

% opposite of above; bump a section back toward the left margin
\newenvironment{unindentsection}[1]%
{\begin{list}{}%
		{\setlength{\leftmargin}{-0.5#1}}%
		\item[]%
	}
	{\end{list}}

% format two pieces of text, one left aligned and one right aligned
\newcommand{\headerrow}[2]
{\begin{tabular*}{\linewidth}{l@{\extracolsep{\fill}}r}
		#1 &
		#2 \\
\end{tabular*}}

% make "C++" look pretty when used in text by touching up the plus signs
\newcommand{\CPP}
{C\nolinebreak[4]\hspace{-.05em}\raisebox{.22ex}{\footnotesize\bf ++}}

% and the actual content starts here
\begin{document}

\vspace*{-1.2cm}
\noindent Jason Bane\\
Newport News, VA \\
\today\\

Dear Professor Charles Cunningham,

\paragraph{}I write to apply for the Assistant Professor for the department of Physics at Grinnell College. I successfully defended my dissertation "The EMC Effect in A=3 Nuclei" under the direction of Nadia Fomin, at the University of Tennessee. Before entering into graduate school, I taught high school physics and mathematics. My dissertation research brought me to Thomas Jefferson National Laboratory in Newport News, VA. My time at the lab has been devoted to honing my skills and knowledge in the complete process of being an experimental physicist. I experienced constructing spectrometers, refurbishing detectors, building electrical systems for data acquisition, analyzing large data sets, and collaborating with a diverse community while working with three electron scattering experimental groups.
\paragraph{}My research at Jefferson Lab and my time as an educator at the secondary and collegiate level has allowed me to develop my skills as a teacher and mentor. Collaborating with a diverse group of scientists to construct, maintain, and operate spectrometers has allowed me to gain a better understanding on how to adapt my skills as a mentor and being a leader to a diverse body of people. Before joining a research group, I worked as a graduate teaching assistant. As the head astronomy lab instructor, I designed lessons that used the planetarium, interactive computer programs, and night time viewing with telescope to facilitate the learning of basic astronomy topics for undergraduate students focusing in variety of majors. My time teaching at the secondary level focused on teaching physics, geometry, and  college prep. math. Teaching these three courses required me to be willing to adopt innovative teaching methods to promote learning for all of my students. I designed lessons that focused on problem solving skills and discussion. I also implemented other teaching techniques appropriate for the concept being taught for example, hands on labs, demonstrations, computer based labs, and student presentations. 
\paragraph{}I believe emphasizes close student-faculty interaction is the path to an inclusive education for all. Not every student can truly learn from a presentation slide. The ability to have quality interactions with all students can help instructors gain insights into the teaching methodologies that will best fit the individual student. These interactions can also benefit the instructor by providing  prospectives outside of the teachers understanding. This can help develop more educational strategies and help the instructor gain a better understanding of diverse backgrounds. 




%My lesson designs use constant assessment through observation and direct questions, while also using self assessment to drive critical thinking. In order to provide the best learning environment, I drive to be compassionate and show respect in all of my interactions inside and out of the classroom in the hope that I might help others embraces diversity and inclusion. 


\paragraph{}The knowledge and skills I have gained while working as a researcher and educator has made me a good candidate for the assistant professor position at 	
Grinnell College. My experiences in the classroom and the laboratory will allow my adapt my lessons to fit the requirements of a diverse group of students. Through undergraduate research programs at both Oak Ridge National Lab and Thomas Jefferson Nation Lab, I can facilitate opportunities for undergraduate students to gain experiences in the fields of nuclear and accelerator physics.  
\\

\noindent Thank you for your time and attention,\\
\noindent Jason Bane




\end{document}

