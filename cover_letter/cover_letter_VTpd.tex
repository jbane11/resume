
\documentclass[12pt,letterpaper]{article}
\usepackage[letterpaper,margin=0.75in]{geometry}
\usepackage[utf8]{inputenc}
\usepackage{mdwlist}
\usepackage[T1]{fontenc}
\usepackage{textcomp}
\usepackage{tgpagella}
\usepackage{multicol}
\usepackage{fancyhdr}

\pagestyle{empty}
\setlength{\tabcolsep}{0em}

\pagestyle{fancy}
\fancyhf{}
\renewcommand{\headrulewidth}{0pt}
\lhead{}
\chead{}
\rhead{}
\rfoot{\thepage}
\lfoot{Jason Bane}
\cfoot{jbane@jlab.org}

% indentsection style, used for sections that aren't already in lists
% that need indentation to the level of all text in the document
\newenvironment{indentsection}[1]%
{\begin{list}{}%
		{\setlength{\leftmargin}{#1}}%
		\item[]%
	}
	{\end{list}}

% opposite of above; bump a section back toward the left margin
\newenvironment{unindentsection}[1]%
{\begin{list}{}%
		{\setlength{\leftmargin}{-0.5#1}}%
		\item[]%
	}
	{\end{list}}

% format two pieces of text, one left aligned and one right aligned
\newcommand{\headerrow}[2]
{\begin{tabular*}{\linewidth}{l@{\extracolsep{\fill}}r}
		#1 &
		#2 \\
\end{tabular*}}

% make "C++" look pretty when used in text by touching up the plus signs
\newcommand{\CPP}
{C\nolinebreak[4]\hspace{-.05em}\raisebox{.22ex}{\footnotesize\bf ++}}

% and the actual content starts here
\begin{document}


{\textbf{Jason Bane}}\\
18 Morrison Ave. \\
Newport News, VA 23601 \\
(931) 239-0611 \\
jbane@jlab.org \\
\today\\

Dear Dr. Camillo Mariani, 

\paragraph{}Completing my research at Jefferson Lab for the last 5 years has afforded me the opportunity to gain an immense amount of experience with electron scattering experiments. My work with the MARATHON, APEX, and Argon($e^\prime$,p) experiments have honed my skills and knowledge in experimental physics. This experience will make me a prime candidate to use data from Jefferson Lab to construct nuclear models. 
\paragraph{}My work at Jefferson Lab has allowed me to work on many aspects of an experiment. I was able to work on the refurbishment and maintenance of the BigBite Spectrometer for the MARATHON experiment. Preparing this spectrometer gave me the opportunity to gain experience in the design and construction of the front end electronics including designing and testing of a logic trigger. I continued my work with BigBite by contributing to the refurbishing of individual detectors in the spectrometer and erecting the data acquisition system. I assisted in the preparation of analysis software for the three different experiments. During the Argon, MARATHON, and APEX experiments, I helped maintain the online analysis software and replay scripts. Using the analysis software and replayed data, I calibrated parts of the detectors, focusing on the beam position monitors and adc signals from the cherenkov and calorimeters. Working with these experiments and their collaborators allowed me to work with large numbers of students, postdocs and other scientists.  Working with such a vast group of scientist developed my skills in communication and collaboration. Being a part of these large collaborations have advanced my abilities in discussion, presenting, and collaborating.
As part of my Ph.D., I have been analyzing MARATHON data. Part of my analysis task has been to compare data results to simulated data. In order to simulate data, I have had to work closely with cross section models and simulation packages. Completing this analysis has granted me the ability to learn different coding languages like \CPP, fortran, ROOT, and python.

\paragraph{}The knowledge I have gained working at Jefferson Lab would make me a great fit for this postdoc at Virginia Tech. My previous work with the Hall A analysis software and programming languages would allow me to quickly analyze Hall A data. My experience in large collaborations will  contribute to supervising and coordinating with other scientists. My familiarity with coding languages and simulation packages will help me develop nuclear models for neutrino interactions.\\

Thank you for your time and attention,

Jason Bane

\newpage








\end{document}